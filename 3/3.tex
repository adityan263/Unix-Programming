\documentclass[main.tex]{subfiles}
\begin{document}
\section{Question 3}

Write a program to include different types of variables to demonstrate the
behavior of setjmp/longjmp.

\begin{itemize}
\item Include a jmp\_buf automatic variable in main()
\item Invoke a function a() with the argument jmp\_buf variable.
\item If return values of a() is nonzero, exit.
\item Then invoke b() with the argument jmp\_buf variable.
\item In a(), return the values of setjmp() invocation with argument as received
jmp\_buf  variable.
\item In b(), just invoke longjum() with  received jmp\_buf  variable and a non zero
value.
\end{itemize}

Describe your observation and understanding

\lstinputlisting[style=codeStyleC]{\asgnno/3.c}

Inside the function a(), the destination for jump is fixed using setjmp and from
the function b(), jump is called using longjmp. Program execution can come out
of setjmp in 2 ways. One is when jump destionation is decided and other is when
longjmp is called. In first case, it returns 0 and in second it returns the
second argument which was passed to longjmp(non zero). Even if we pass zero as
second argument to longjmp, it will still return 1. After longjmp, execution
continues from the setjmp line, which can be seen in following image as "Entered
a()" is printed only once.

Optimizations don’t affect the global, static, and volatile variables; their
values after the longjmp remain same. Without optimization, global, static,
auto, register and volatile variables are stored in memory (the register hint is
ignored for regival). When we enable optimization, both autoval and regival go
into registers, even though the former wasn’t declared register, and the
volatile variable stays in memory. The important thing to realize is that we
must use the volatile attribute if we are writing portable code that uses
nonlocal jumps. Anything else can change from one system to the next.

The setjmp(3) manual page on one system states that variables stored in memory
will have values as of the time of the longjmp, whereas variables in the CPU and
floating-point registers are restored to their values when setjmp was called.

\centering\includegraphics[width=\linewidth]{\asgnno/3.png}
\clearpage
\end{document}

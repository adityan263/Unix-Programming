\documentclass[main.tex]{subfiles}
\begin{document}
\section{Question 4}

Print all existing environment variables with their values. Later input a new
variable and its value and add to the environment list. Once again print the
list.

\lstinputlisting[style=codeStyleC]{\asgnno/4.c}

In this program, 2 different functions are used to add new environment variables
and their value. 1. putenv  2. setenv

In putenv, we have to provide an equation with variable name on LHS and value on
RHS, then the function adds the environment variable. If that variable already
exists, then it will replace its value with the new provided value(difference
between 1.txt and 2.txt is that new variable is added in later one).

In setenv, it has 3 arguments. First is name of variable, second is its value
and third is replacement value. If the variable doesn't exist, then it is added
and the new value is assigned to it. In this case value of replacement value
doesn't matter(difference between 1.txt and 3.txt is that new variable is added
in later one).
If variable exists and the replacement value is zero, then value of variable is
kept as it is(so 1.txt and 4.txt are same).
If variable exists and replacement value is non zero, then the new value(second
argument) is assigned to the environment variable(difference between 4.txt and
5.txt is that value of the first new variable is changed).

\centering\includegraphics[width=\linewidth]{\asgnno/4.png}
\clearpage
\end{document}

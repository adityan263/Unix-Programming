\documentclass[main.tex]{subfiles}
\begin{document}
\section{Question 1}

Study syslogd to understand how a daemon can able to display error messages by
writing to a file. Prepare the important points and code snippets of syslog
facility.


\section*{Answer}

A daemon runs in the background, we can’t display messages on terminal, as we
would do with other programs. So we write messages to an application-specific
log file.

The syslog facility provides a single, centralized logging facility that can be
used to log messages by all applications on the system. The syslog facility has
two principal components: the syslogd daemon and the syslog library function.

\subsection{syslogd}

The System Log daemon, syslogd, accepts log messages from two different sources:
a UNIX domain socket, /dev/log , which holds locally produced messages, and an
Internet domain socket, which holds messages sent across a TCP/IP network.

Each message processed by syslogd has a number of attributes, like facility,
which specifies the type of program generating the message, and a level, which
specifies the severity of the message. The syslogd daemon examines the facility
and level of each message, and then passes it to any of several possible
destinations according to the dictates of an associated configuration file,
/etc/syslog.conf . Possible destinations include a terminal or virtual console,
disk file and/or FIFO.

\subsection{syslog API}

The API consists of 3 main functions:

\subsubsection{openlog()}
\lstinputlisting[style=codeStyleC]{\asgnno/openlog.h}

Fist argument, indent, is a char pointer which denotes a string which is written
in every log of this program. log\_option is a bitmask created by ORing together
uwith any of constants like LOG\_CONS, LOG\_NDELAY, LOG\_NOWAIT etc. This
argument provides the default facility for subsequent syslog() call.

\subsubsection{syslog()}
\lstinputlisting[style=codeStyleC]{\asgnno/syslog.h}

The priority argument is created by ORing together a facility value and a level
value. The facility indicates the general category of the application logging
the message.  The rest of the arguments work like printf that is by providing
string format first and then rest of the values in the string.

\subsubsection{closelog()}
\lstinputlisting[style=codeStyleC]{\asgnno/closelog.h}

When we have finished logging, we can call closelog() to deallocate the file
descriptor used for the /dev/log socket. Since a daemon typically keeps a
connection open to the system log continuously, it is common to omit calling
closelog().

\vskip 0.3in

Using setlogmask() function, we can create a mask for specific priority levels
to be filtered. Using LOG\_UPTO() Macro we can bit mask filtering all messages
of a certain level and above. It returns previous log priority mask.


\lstinputlisting[style=codeStyleC]{\asgnno/1.c}
\includegraphics[width=\linewidth]{\asgnno/1.png}

In the code, initially log mask is set such that only the logs with priority
lower than or equal to LOG\_NOTICE will be printed. It can be seen in the output
that only the first log in printed in the file because of the mask. None of
these functions returns a status value. This is because system logging should
always be available.

\clearpage
\end{document}

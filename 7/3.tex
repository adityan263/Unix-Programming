\documentclass[main.tex]{subfiles}
\begin{document}
\section{Question 3}

Implement sleep using signal function which takes care of the following:
\begin{enumerate}
	\item If the caller has already an alarm set, that alarm is not erased
		by the call to alarm inside sleep implementation.
	\item If sleep modifies the current disposition of SIGALRM, restore it.
	\item Avoid race condition between first call to alarm and pause inside
		sleep implementation using setjmp.
\end{enumerate}
Test the implementation of sleep by invoking it in various situations.

\lstinputlisting[style=codeStyleC]{3.c}

The new\_sleep function changes the signal handler for SIGALRM and remembers the
old one. Then it sets alarm for given time and calls pause(). As the alarm
function retuns the remaining time of previous alarm, we remember it for future.
Between alarm(n) and pause() it sends SIGINT to itself just to test what happens
in case of race condition. After completion of pause(), old disposition is set
for signal again.  As the signal gets masked by signal() after calling signal
handler, we have to unmask the signal, calls alarm function for remembered alarm
time and exits.

\includegraphics[width=\linewidth]{3.png}

To test whether previously set alarm and the old disposition are remembered or
not, an alarm and a disposition are set before calling new\_sleep and pause is
called at the end.

As SIGINT is sent between alarm() and pause() call, race condition can be
observed. Still the process doesn't go in infinite loop, this shows that race
condition is handled properly.

When the code comes out of new\_sleep, the process comes out of pause() because
previous set alarm is not discarded by the new\_sleep function and the interrupt
also calls the old signal handler, thus old alarm and disposition are not
discarded.

\clearpage
\end{document}

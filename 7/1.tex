\documentclass[main.tex]{subfiles}
\begin{document}
\section{Question 1}

Catch the SIGTERM signal, ignore SIGINT and accept the default action for
SIGSEGV. Later let the program be suspended until it is interrupted by a signal.
Implement using signal and sigaction.

\subsection{Using signal()}
\lstinputlisting[style=codeStyleC]{1a.c}

Signal function takes two arguments, signal name and handler function. SIGTERM
is caught using sigterm\_handler(). To ignore or set default action for signals,
SIG\_IGN or SIG\_DFL are passed as second arguments.

\includegraphics[width=\linewidth]{1a.png}


\subsection{Using sigaction()}
\lstinputlisting[style=codeStyleC]{1b.c}

struct signaction action is declared and all of its variables are set to NULL or
0, except the sa\_handler(signal handler). Just by changing sa\_handler, same
action struct is used for setting handlers for different signals.

\includegraphics[width=\linewidth]{1b.png}

As we are allowing default action to take place when SIGSEGV is received, the
process/code stops executing. On sending SIGINT to process, it continues
execution as if nothing has happened.

\clearpage
\end{document}

\documentclass[main.tex]{subfiles}
\begin{document}
\section{Question 5}

Describe various features of IPC mechanisms.


\subsection{Pipes:}

Pipes are used for communication between 2 processes e.g. ls | grep vim
Here standard output of ls command is directly fed to the standard input of second
command. Then the output of second command is displayed.
Neither of the processes is aware of the other one, shell creates a pipe and
redirects the output. For reading and writing into file, standard file library
function are used. There are special pipes also known as named pipes. FIFO is
one of them.

\subsection{Message Queues:}

In message queues one or more that one processes can read from them and multiple
processes can write into them.	While reading or writing, access rights of the
executing process are checked before granting permission. POSIX message queues
allow message notification option.

\subsection{Semaphore:}

Semaphore is a memory location which can be accessed by multiple processes at
the same time. For writing, each process has to call test and set function, this
is done to avoid multiple processes writing at the same time. Test function
returns appropriate value which tells if any other process is writing at that
time or not.

\subsection{Shared Memory:}

Shared memory allows one or more processes to communicate via memory that
appears in all of their virtual address spaces. The pages of the virtual memory
is referenced by page table entries in each of the sharing processes' page
tables. Access to shared memory is given by checking keys and access rights.
Shared memory is a better option than others as it is very efficient and it also
allows random access to memory locations. Efficiency is more because unlike
message queues and pipes, which copy data from the process into memory within
the kernel, shared memory is directly accessed.

\clearpage
\end{document}
